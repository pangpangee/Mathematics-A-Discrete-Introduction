\documentclass{article}
\usepackage{ amssymb }
\usepackage{amsmath}
\usepackage{hyperref}
\hypersetup{
    colorlinks=true,
    linkcolor=blue,
    filecolor=magenta,      
    urlcolor=cyan,
    pdftitle={Overleaf Example},
    pdfpagemode=FullScreen,
    }
\usepackage[shortlabels]{enumitem}
\title{Mathematics: A Discrete Introduction}
\author{Written by Edward Scheinerman\\Answer sheet by Kim Taewon\\Pukyong National University\\Department of Computer Engineering}
\date{\today}

\begin{document}

\maketitle

\section*{$\sharp$10: Quantifiers}
\subsection*{10.1}

\begin{enumerate}[a.]
    \item $\forall x \in \mathbb{Z}, x \textrm{ is prime}.$
    \item $\exists x \in \mathbb{Z}, \neg(x\textrm{ is prime or composite}).$
    \item $\exists x \in \mathbb{Z}, x^2=2.$
    \item $\forall x \in \mathbb{Z}, 5|x.$
    \item $\exists x \in \mathbb{Z}, 7|x.$
    \item $\forall x \in \mathbb{Z}, x^2 > 0.$
    \item $\forall x \in \mathbb{Z}, \exists y \in \mathbb{Z}, xy = 1.$
    \item $\exists x,y \in \mathbb{Z}, x/y = 10.$
    \item $\exists x \in \mathbb{Z}, \forall y \in \mathbb{Z}, xy = 0.$
    \item $\forall x \in \mathbb{Z}, \exists y \in \mathbb{Z}, x < y.$
\end{enumerate}

\subsection*{10.2}
\begin{enumerate}[a.]
    \item There is an integer that is not prime.
    \\$\exists x \in \mathbb{Z}, x \textrm{ is not prime}.$
    \item Every integer is prime or composite.
    \\$\forall x \in \mathbb{Z}, x\textrm{ is prime or composite}.$
    \item Every integer is not an integer whose square is $2$.
    \\$\forall x \in \mathbb{Z}, \neg(x^2=2).$
    \item There is an integer that is not divisible by $5$.
    \\$\exists x \in \mathbb{Z}, \neg(5|x).$
    \item All integers are not divisible by $7$.
    \\$\forall x \in \mathbb{Z}, \neg(7|x).$
    \item There is an integer whose square is negative.
    \\$\exists x \in \mathbb{Z}, x^2 < 0.$
    \item There is an integer $x$ for any integer $y$ such that $xy\neq1$.
    \\$\exists x \in \mathbb{Z}, \forall y \in \mathbb{Z}, xy \neq 1.$
    \item For every integer $x,y$, $x/y\neq=10.$
    \\$\forall x,y \in \mathbb{Z}, x/y \neq 10.$
    \item For every integer $x$ there is an integer $y$ such that $xy\neq0$.
    \\$\forall x \in \mathbb{Z}, \exists y \in \mathbb{Z}, xy \neq 0.$
    \item There is an integer $x$ which is equal or greater than any integer.
    \\$\exists x \in \mathbb{Z}, \forall y \in \mathbb{Z}, x \geq y.$
\end{enumerate}

\subsection*{10.3}
No one is invited to the party.

\subsection*{10.4}
\begin{enumerate}[a.]
    \item False
    \item True
    \item True
    \item True
    \item False
    \item True
    \item True
    \item True
\end{enumerate}

\subsection*{10.5}
\begin{enumerate}[a.]
    \item $\exists x \in \mathbb{Z}, x \geq 0.$
    \\There is an integer that is greater than or equal to $0$.
    \item $\forall x \in \mathbb{Z}, x \neq x+1.$
    \\Every integer $x$ is not equal to $x+1$.
    \item $\forall x \in \mathbb{Z}, x \leq 10.$
    \\All integers are less or equal to $10$.
    \item $\exists x \in \mathbb{Z}, x + x \neq 2x.$
    \\There is an integer $x$ such that $x + x \neq 2x.$
    \item $\forall x \in \mathbb{Z}, \exists y \in \mathbb{Z}, x \leq y.$
    \\For every integer $x$ there is an integer $y$ such that $x \leq y$.
    \item $\exists x \in \mathbb{Z}, \exists y \in \mathbb{Z}, x \neq y.$
    \\There is an integer $x$ such that there is an integer $y$ which satisfies $x \neq y$.
    \item $\exists x \in \mathbb{Z}, \forall y \in \mathbb{Z}, x + y \neq 0.$
    \\There is an integer $x$ for any integer $y$ such that $x+y\neq0$.
\end{enumerate}

\subsection*{10.6}
$$\begin{aligned}
&\forall x, \forall y, \textrm{assertions about }x\textrm{ and }y\\
&\forall y, \forall x, \textrm{assertions about }x\textrm{ and }y\end{aligned}$$

These two statements mean the same thing, since we have "uninterrupted subsequences of quantifiers of the same type". Likewise, the following statements mean the same thing.

$$\begin{aligned}
&\exists x, \exists y, \textrm{assertions about }x\textrm{ and }y\\
&\exists y, \exists x, \textrm{assertions about }x\textrm{ and }y\end{aligned}$$

"Being able to swap the use of the variables is of course due to the fact that variables are just 'dummies', and can therefore be replaced by other variables." Check: \url{https://math.stackexchange.com/q/3515811}.
\end{document}
